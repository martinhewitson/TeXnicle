% DOCUMENTO PRINCIPALE DELLA TESI
% Leggere i commenti nel file e i file PDF acclusi
%
% Ultima modifica: 10/10/2007

\documentclass[nocite]{cls/dimeg04}
%Le opzioni possibili sono:
%  nocorr 	:non mette il correlatore
%  nofront	:non mette il frontespizio (col logo unipd)
%  nolatex	:non mette il retro-frontespizio (coll'indicazione del latex)
%  nodedica	:non mette la pagina con la dedica
%  nocite 	:non mette la pagina con la citazione

\usepackage{latexsym}

%per rendere l'indice cliccabile
%\usepackage[dvips]{hyperref}
\usepackage[pdftex]{hyperref}
%\usepackage[ps2pdf,colorlinks]{hyperref}. 

%per poter inserire le figure
%\usepackage[dvips]{graphicx}
\usepackage[pdftex]{graphicx}

%per suddividere le figure in sottofigure
%\usepackage{subfigure}

%titoli e sillabazione in italiano
\usepackage[italian]{babel}

%per usare lettere greche in grassetto (inserito da Oscari)
\usepackage{bm}

%per allineare equazioni spezzate, etc. (inserito da Oscari)
\usepackage{amsmath}

%per figure allineate con 1 didascalia (inserito da Oscari)
\usepackage{subfig}

%creato da Oscari per pedici/apici senza math mode
\newcommand{\superscript}[1]{\ensuremath{^{\textrm{#1}}}}
\newcommand{\subscript}[1]{\ensuremath{_{\textrm{#1}}}}

%per simbolo dell'euro (inserito da Oscari)
%\usepackage{eurosym}
%\usepackage{textcomp}
\usepackage{marvosym}

%per forzare la posizione dell figure (inserito da Oscari)
\usepackage{float}

%per ruotare tabelle (inserito da Oscari)
%\usepackage{lscape}

%per andare a capo nelle celle delle tabelle (inserito da Oscari)
\usepackage{tabularx}

%per celle multiriga nelle tabelle (inserito da Oscari)
\usepackage{multirow}

%per ombre oggetti (inserito da Oscari)
%\usepackage{pstricks}

%correggere manualmente la sillabazione delle parole che il babel non sillaba correttamente
\hyphenation{ro-bo-ti-ca me-di-ca}
\hyphenation{ge-ne-ra-li}
\hyphenation{ca-rat-te-ri-sti-che}
\hyphenation{ci-ne-ma-ti-ca}


%per poter usare l'ambiente verbatim (va benissimo per il codice sorgente)
\usepackage{verbatim}

%pacchetto floatflt.sty' che permette di avere figure a lato del testo 
\usepackage[vflt]{floatflt}


%regolazioni pagina
\linespread{1.4} %interlinea
\voffset-0.3cm   %'1 pollice + voffset' é la distanza tra la pagina e il bordo alto


\frenchspacing
\begin{document}


%qui vanno specificati i dati della Tesi (rispettare Maiuscolo-Minuscolo !)
\author{FABIO OSCARI}
\title{STUDIO DI SISTEMI ROBOTICI ATTIVI INTERAGENTI PER RIEDUCAZIONE MOTORIA}
\relatore{Ch.mo Prof. Ing. ALDO ROSSI} 
\correlatore{Ch.mo Prof. Ing. ALBERTO TREVISANI}
\annoa{XIV}

% utilizzare la prima riga per tesi Specialistiche, la seconda per Triennali
\tipoLaurea{INDIRIZZO MECCATRONICA E SISTEMI INDUSTRIALI}

\dedica{alla mia famiglia e alla mia ragazza...}
%\citazione{\textquotedblleft{} Ora ti dirň tutta la mia idea intorno alla Casa dello Specchio, bla bla \textquotedblright}
%\autoreCitazione{Lewis Carroll - Alice attraverso lo specchio, 1871}


%Inizio della tesi, numerazione latina, e indice
\pagenumbering{Roman}
\maketitle
\tableofcontents

%\include{Sommario}
%\include{Introduzione}

%Inizio del corpo della tesi, numerazione araba
\pagestyle{headings}
\setcounter{chapter}{0}\setcounter{section}{0}
\pagenumbering{arabic}

%Inclusione dei vari capitoli
\part{Neuroriabilitazione della mano}
%Descrizione della parte 1
Nel primo capitolo viene focalizzata l'attenzione sul problema dell'ictus, dal punto di vista clinico ed epidemiologico, insieme agli obiettivi e le modalit� della riabilitazione neurologica.

Nel secondo capitolo l'attenzione si focalizza sulla riabilitazione della mano, descrivendo in particolare l'anatomia della mano da un lato e il sistema di controllo del movimento dall'altro.

Nel terzo capitolo vengono menzionati i campi d'applicazione della robotica con un breve confronto rivolto al campo terapeutico o riabilitativo, evidenziando anche le sostanziali differenze tra robot per usi riabilitativi rispetto ai robot tradizionali, utilizzati per impieghi industriali. Si descrive inoltre un'interfaccia molto utilizzata nei robot riabilitativi qual � quella aptica.

Infine nel capitolo quattro sono elencati e descritti i principali robot per la riabilitazione della mano con lo scopo di fornire una panoramica generale sulla situazione mondiale in questo settore.

\chapter{L'ictus e la riabilitazione post-stroke}
\label{cha:cap1}

\section{La riabilitazione e l'ictus}
%
% aggiungere prima un'introduzione sulla riabilitazione in generale

% fonte: Andreolli (eventualmente Cris) magari anche Tessarolo ---> CRIS DEVE MODIFICARE
L'ictus, o stroke nell'accezione americana, � una patologia di forte impatto nella nostra societ� sia per l'elevata incidenza e mortalit�, sia per l'alto numero di soggetti che presentano sequele disabilitanti. I costi per l'assistenza sanitaria e le cure ospedaliere sono enormi ed a ci� si aggiunge il peso che grava sulle famiglie e la perdita di produttivit� lavorativa.\\
Dalla letteratura emerge che nei pazienti colpiti da ictus il trattamento riabilitativo � tanto pi� efficace quanto pi� risulta precoce, intensivo e ricco di stimoli multisensoriali. Per cercare di soddisfare queste esigenze i pazienti necessitano di essere sottoposti ad un approccio fisioterapico individuale (hand to hand) di almeno 3 ore al d�, che spesso non risulta praticabile nelle nostre strutture medico-riabilitative.\\
Nasce da tali considerazioni l'idea (e la necessit�) di ricercare nuove tecniche e nuovi strumenti da affiancare al lavoro del fisioterapista per rispondere in modo adeguato ed efficace alle esigenze terapeutiche di questi pazienti, con l'obiettivo anche di un contenimento dei costi.

\subsection{Clinica e valutazione dello stroke}
%
% fonte: Andreolli (eventualmente Cris e sicuramente me lo deve controllare) ---> CRIS DEVE MODIFICARE

L'\emph{ictus} � la manifestazione clinica di una malattia cerebro-vascolare caratterizzata dall'``improvvisa comparsa di segni e/o sintomi riferibili a deficit focale e/o globale (stato di coma) delle funzioni cerebrali, di durata superiore alle 24 ore o ad esito infausto'' \cite{AV05}. Tale definizione comprende l'infarto ischemico, l'emorragia intracerebrale primaria e alcuni casi di emorragia subaracnoidea. Il quadro clinico di un ictus varia a seconda della zona cerebrale interessata, del grado di risparmio dei circoli collaterali e dall'entit� dell'evento. Si distinguono tre forme di ischemia cerebrale: 
\begin{enumerate}
	\item TIA (Transient Ischemic Attacks), con regressione completa della sintomatologia o del deficit neurologico in meno di 24 ore;
	\item RIND (Reversible Ischemic Neurologic Deficit), con regressione completa della sintomatologia in qualche giorno;
	\item Ictus, con esito mortale o con persistenza dei difetti neurologici focali.
\end{enumerate}

Tra i fattori clinici, la gravit� di presentazione dell'ictus rappresenta un importante fattore predittivo dell'esito funzionale a medio-lungo termine. Molte scale sono state sviluppate in questi anni con lo scopo di misurare tale gravit� e per monitorarne l'evoluzione nel tempo, anche ai fini della definizione prognostica e del tipo di intervento riabilitativo da effettuare. Sono qui citate le pi� utilizzate nella valutazione della forza e della potenza muscolare, della motricit�, della disabilit� e indipendenza funzionale e del grado di deficit motorio.\\
Per la valutazione clinica della forza e della potenza muscolare viene solitamente utilizzata la MRC Scale (Scala delle paresi secondo il Medical Research Council). La
MRC � un test che permette di valutare il grado di compromissione motoria di singoli muscoli o di gruppi di muscoli; infatti si passa da un grado 0, in cui non � visibile o palpabile nessun movimento, fino ad un grado 5, in cui il paziente � in grado di produrre un movimento contro la resistenza dell'esaminatore \cite{Mas99}.\\
Il Motricity Index � un test che valuta le abilit� motorie a livello dell'arto superiore e dell'arto inferiore. Per la corretta somministrazione della scala il paziente viene solitamente seduto su una sedia oppure sul bordo del letto, ma all'occorrenza pu� essere valutato anche da sdraiato: la valutazione include il test di sei movimenti, di cui tre per l'arto superiore e tre per l'arto inferiore \cite{Mas99}.\\
Tra le scale motorie pi� frequentemente utilizzate vi � anche la scala di FuglMeyer che considera 4 ambiti di prestazione: il livello di coscienza, la prestazione
motoria, la comunicazione verbale e la capacit� di percezione. La valutazione della prestazione motoria di un arto comprende tre parti differenti ma interdipendenti: la
funzione motoria e l'equilibrio, la qualit� della sensibilit�, il range di movimento passivo e l'eventuale presenza di dolore alle articolazioni \cite{Mas99}.\\
La Scala di Barthel (o Barthel Index) rappresenta una modalit� di misura universalmente accettata per la valutazione del grado di autonomia e disabilit�; presenta un'elevata ripetibilit� ed affidabilit� che rende appropriata la scala anche per il monitoraggio e per la prognosi funzionale del paziente con lesione cerebrovascolare. Richiede pochi minuti di osservazione del paziente da parte dell'operatore (medico o non) ed esplora 10 item pesati concernenti le principali attivit� della vita quotidiana (mangiare, lavarsi, vestirsi, spostarsi dalla sedia al letto, mobilit�, capacit� di salire le scale etc).\\
Anche la Functional Independence Measure (FIM) � una scala di valutazione della disabilit� frequentemente utilizzata; come il Barthel Index anch'essa � utilizza-
ta per valutare il grado di assistenza richiesti nel compiere le normali attivit� quotidiane. Si considerano item funzionali (cura della persona, trasferimentii, controllo sfinterico, ecc.), ma a differenza della Barthel Index vengono inclusi anche 5 item attinenti la valutazione degli aspetti cognitivi (comunicazione, memoria, giudizio, ecc.) \cite{Vic02} \cite{AV05}.\\
Tra le scale di valutazione del deficit neurologico, in genere molto pi� accurate nella valutazione degli ictus di gravit� media rispetto a quelli di gravit� maggiore, la National Institutes of Health Stroke Scale (NIHSS) � la scala pi� frequentemente utilizzata; comprende 15 item che esplorano lo stato di coscienza, la visione, i movimenti extraoculari, la paralisi del facciale, la forza degli arti, l'atassia, la sensibilit�, la parola ed il linguaggio. Questa scala si presta bene ad effettuare valutazioni seriate e pertanto trova indicazione clinica per quantificare miglioramenti o peggioramenti del quadro neurologico durante il follow-up \cite{Vic02}.


\subsection{Epidemiologia dell'ictus}
%
% fonte: Andreolli ---> CRIS DEVE MODIFICARE

Nei paesi industrializzati tra cui l'Italia l'ictus � la terza causa di morte dopo le malattie cardiovascolari e le neoplasie, essendo responsabile del 10-12\% di tutti i decessi per anno (circa 400.000) nei paesi della CEE. Un recente studio europeo asserisce che ci sono circa 250-350 casi di stroke all'anno ogni 100.000 persone. E si ipotizza che con l'aumento dell'et� media, e conseguentemente del numero di anziani, con i cambiamenti demografici e con la persistente abitudine al fumo, tale numero potrebbe raddoppiare entro il 2025 \cite{2006_Truelsen}. Tale patologia riveste un'enorme importanza sociosanitaria per un duplice motivo: quello della mortalit� che rimane elevata e quello della disabilit� residua, grave fonte di sofferenze e di costi per i malati, per le loro famiglie e per la collettivit�. L'ictus cerebrale � riconosciuto come la patologia pi� costosa in termini di qualit� della vita che monetari (costi assistenziali e terapeutici ospedalieri ed extraospedalieri, costi indiretti dovuti alla perdita di produttivit� per assenze dal lavoro, pensionamento anticipato, ecc. e costi familiari).

Per quanto riguarda l'Italia i dati nazionali di prevalenza e di incidenza sono quelli dello studio ILSA (Italian Longitudinal Study on Aging): nella popolazione anziana (65-84 anni) italiana il tasso di prevalenza � pari a 6,5\%; lievemente superiore negli uomini (7,4\%) rispetto alle donne (5,9\%). L'incidenza aumenta progressivamente con l'et� raggiungendo il massimo negli ultraottantacinquenni. Da questi studi di popolazione risulta che circa l'80\% dei soggetti con ictus � affetto da forme di tipo ischemico, mentre le emorragie intraparenchimali rappresentano circa il 15-20\% dei casi e le emorragie subaracnoidee non superano il 3\% del totale. L'ictus ischemico colpisce soggetti con et� media superiore a 70 anni, pi� spesso uomini che donne; quello emorragico intraparenchimale colpisce soggetti leggermente meno anziani, sempre con lieve prevalenza per il sesso maschile, mentre l'emorragia subaracnoidea colpisce pi� spesso soggetti di sesso femminile di et� media intorno ai 45-50 anni. Ogni anno in Italia vi sono, quindi, 196.000 nuovi casi di stroke, di cui l'80\% � rappresentato da nuovi episodi (157.000) e il resto da recidive (39.000). Inoltre di questi 196.000 nuovi casi, una minoranza (circa il 20\%) decede nel primo mese successivo all'evento e circa il 30\% sopravvive con esiti
gravemente invalidanti. Sulla base di uno studio della popolazione italiana del 2001 sono risultati 913.000 soggetti sopravvissuti ad un episodio di ictus con esiti pi� o meno disabilitanti. Lo studio ILSA riporta che nei sopravvissuti la disabilit� in almeno una delle ADL (Activities of Daily Living) � presente nel 67,6\% dei pazienti colpiti da un primo evento ictale, e nel 35\% di questi pazienti, ad un anno dall'evento acuto, permena ancora una grave disabilit� (motoria e funzionale), e una importante limitazione nelle normali attivit� della vita quotidiana tanto da poterli definire ``dipendenti'' o ``non autosufficienti'' \cite{AV05} \cite{2003_DiCarlo}.

Nella Regione Veneto, nel 1999, sono stati stimati quasi 10.000 casi di ictus: l'attack rate complessivo e la mortalit� sono, rispettivamente, 219 e 51,8 per 100.000
abitanti/anno. Nei giovani prevalgono le forme emorragiche, mentre con l'incremento dell'et� cresce la percentuale di ictus ischemici. L'et� media dei pazienti con ictus risulta essere di 74 anni ma � diversa a seconda del tipo di ictus. Complessivamente si stima che circa il 73\% siano ictus ischemici, 15\% emorragie cerebrali, il 4\% emorragie subaracnoidee e l'8\% ictus non specificati. La mortalit� interessa il 24\% del totale, pi� elevata nei pazienti che presentano un ictus di tipo emorragico, mentre le recidive rappresentano circa il 23\% del totale degli accidenti cerebro-vascolari per anno \cite{2004_Vari} \cite{2005_Spolaore}.

Nell'anno 2005 si sono registrati a Padova pi� di 2.200 ictus: l'et� media dei colpiti � di 73,8 anni mentre la distribuzione per tipologia dimostra che il 75,5\% sono ictus ischemici e il 14,5\% emorragici concordando con i dati regionali e nazionali. Tra i sopravvissuti, all'atto della dimissione, il 75\% dei pazienti torna al proprio domicilio presentando una disabilit� di grado lieve che comunque nella maggioranza dei casi richiede un trattamento in regime ambulatoriale, mentre il restante 25\% richiede un trattamento riabilitativo specifico in regime di degenza ospedaliera (presso un centro di riabilitazione) o extraospedaliera, presso una Residenza Sanitaria Assistenziale o altro istituto di riabilitazione poich� presenta una disabilit� di grado medio-elevato. Sulla base dei dati ottenuti dalle SDO risulta che a Padova circa 200-250 pazienti ogni anno presentano una disabilit� post-stroke tale da richiedere interventi riabilitativi in regime di ricovero. A questa percentuale vanno aggiunti anche i pazienti con grave disabilit� cronica che annualmente ripetono un ciclo di trattamento riabilitativo in regime di ricovero presso un Centro di riabilitazione che riguarda una percentuale intorno al 10-15\% ossia circa 150-200 soggetti/anno.


\subsection{Trattamento riabilitativo}
%
% fonte: Andreolli (eventualmente Cris) ---> CRIS DEVE MODIFICARE (AGGIUNGERE TERAPIA MENTALE)

Il trattamento riabilitativo del paziente con esiti di lesione cerebrovascolare dovrebbe essere intrapreso il pi� precocemente possibile abbinandolo alla terapia farmacologia con lo scopo di favorire in massima misura il recupero delle funzioni perse e, quindi, permettere il reinserimento e l'indipendenza nell'ambiente socio-familiare \cite{Mas99}. La terapia riabilitativa � principalmente diretta a favorire modificazioni funzionali e adattamenti cerebrali che rientrano nell'ambito della ``plasticit� cerebrale'' come evidenziato con tecniche di imaging funzionale. Da recenti studi emerge che nei pazienti con esiti di stroke, quanto pi� precocemente s'interviene con la riabilitazione tanto maggiore sar� il recupero motorio e funzionale che si potr� ottenere. Inoltre un trattamento riabilitativo
risulta tanto pi� efficace quanto pi� esso � ricco di stimoli multisensoriali, somministrati precocemente e prolungatamente nel tempo.

Tipicamente i soggetti emiplegici colpiti da ictus necessitano di un approccio fisioterapico individuale ``hand-to-hand'' che nella fase intensiva non dovrebbe essere inferiore a 3 ore al giorno. Nei pazienti con ictus � dunque indicato attivare fin dalla fase acuta un intervento riabilitativo con il programma diagnostico e terapeutico di emergenza. Scopo dell'intervento riabilitativo � quello, oltre che di prevenire le complicanze legate all'immobilizzazione durante la fase acuta, di stimolare il paziente all'apprendimento di nuove abilit� motorio-funzionali sfruttando tutti i sistemi funzionali rimasti integri, al fine di ottenere la maggior indipendenza possibile nella vita quotidiana. A questo obiettivo concorrono sia strategie mirate a ridurre il deficit motorio e cognitivo, sia tecniche di addestramento a comportamenti compensatori, che garantiscono il perseguimento di un'indipendenza funzionale nonostante la persistenza delle menomazioni.

La perdita di abilit� nell'uso dell'arto superiore rappresenta forse la principale causa di disabilit� nei pazienti con lesione cerebrovascolare. Si stima che circa il 20\% dei soggetti non recuperi nessun uso funzionale dell'arto superiore e che l'85\% vada incontro ad un recupero parziale \cite{1992_Gowland}. Questa condizione non necessariamente contrasta con l'acquisizione di un buon livello d'autonomia, ma sicuramente penalizza il recupero dell'attivit� professionale e il reinserimento sociale \cite{2002_Lai}. Il recupero funzionale dell'arto superiore rappresenta un obiettivo a breve e medio termine del progetto riabilitativo per cui � sempre indicato attivare precocemente uno specifico programma di riabilitazione per il recupero dell'arto superiore paretico/plegico. Per la riabilitazione si privilegiano tecniche riabilitative di stimolazione sensitivo-motoria, bench� l'evidenza a supporto dei singoli approcci sia ancora scarsa. Maggiori vantaggi in pazienti selezionati sono stati riportati con l'utilizzo di approcci che comportano una mobilizzazione intensiva (spesso ripetitiva) come ad esempio nelle tecniche che comportano un ``uso forzato dell'arto leso indotto da immobilizzazione dell'arto sano'' (CIMT) \cite{AV05}.

La letteratura riporta che un intervento riabilitativo intensivo nei primi 3 mesi dopo l'evento acuto determini un outcome motorio e funzionale migliore e pi� stabile nel tempo. Nelle attuali realt� ospedaliere italiane un intervento riabilitativo intensivo (cio� per un tempo di 3 ore al d�) � difficilmente praticabile poich� gli elevati costi e l'organizzazione dei tempi costringono i fisioterapisti a mediare le esigenze delle strutture con quelle dei pazienti. Le croniche carenze di personale dei Centri di Riabilitazione rendono purtroppo solo parziali gli ``approcci intensivi'' con risultati che ricadono sul paziente (riducendo la quantit� di riabilitazione somministrata, i recuperi motorio-funzionali sono minori) e sulla societ� (i pazienti presentano un grado di disabilit� maggiore di cui deve farsi carico la societ�).

Nasce da queste considerazioni l'idea di costruire delle macchine robotizzate con l'obiettivo, per l'appunto, di supportare il lavoro del fisioterapista, di aumentare l'intensit� delle terapie somministrate e, soprattutto, di ricercare un contenimento dei costi di trattamento. L'obiettivo finale � quello di ridurre il pi� possibile il grado disabilit� del paziente con lesione cerebrovascolare e, conseguentemente, permettere il pi� precocemente e il pi� completamente possibile il reinserimento socio-familiare.

Dall'inizio degli anni Novanta alcuni Centri di ricerca internazionali stanno sperimentando macchine robotizzate da impiegare in Neuroriabilitazione; i risultati ottenuti nei pazienti che hanno sperimentato questo approccio hanno dimostrato un aumento di forza muscolare dell'arto superiore paretico, una riduzione del deficit motorio e un miglioramento della funzionalit� dell'arto. Il miglior outcome motorio e funzionale ottenuto � stato dimostrato sia a breve che a lungo termine. La stimolazione senso-motoria ottenuta dal training con questi apparecchi robotici permette al soggetto di ricevere
impulsi selettivi e intensivi i quali, probabilmente, stimolando la corteccia cerebrale sia nelle zone limitrofe alla lesione, che in altre aree cerebrali deputate al movimento, determinano risultati motori e funzionali superiori alla sola tradizionale riabilitazione. Le conseguenti modificazioni dell'attivit� cerebrale (che vanno sotto il nome di plasticit� cerebrale) indotte dal training robotico sarebbero la principale causa dei miglioramenti funzionali e motori ottenuti dal paziente.

In conclusione, sulla base dei risultati clinici ottenuti dalla sperimentazione, i medici affermano come quella della riabilitazione mediante apparecchi robotici rappresenti una nuova via nei programmi di Neuroriabilitazione \cite{Mas06a}.



%\include{Cap2/capitolo2}
%\include{Cap3/capitolo3}
%\include{Cap4/capitolo4}

\part{Progettazione e validazione sperimentale di architettura di controllo per tele-valutazione e tele-riabilitazione della mano}

%\include{Cap5/capitolo5}
%\include{Cap6/capitolo6}

\part{Progettazione e realizzazione di nuovo prototipo di ortesi motorizzata}

\part{Controllo multi-feedback con sensory substitution}

%Questi sono capitoli particolari, senza la scritta "capitolo"
%\include{Conclusioni}
%\include{Appendice/Appendice}
%
%\include{Bibliografia}
%\include{Grazie}

\end{document}
