%%!TEX TS-program = Lilypond-XeLaTeX
\documentclass[BCOR12mm,DIV11,headings=normal]{scrbook}
\usepackage{scrpage2}\synctex=1\usepackage{graphicx}\graphicspath{{/Users/jurgenkramlofsky/Desktop/Das-Buch/Bilder2/}}
\usepackage{appendix}%%%% Andere Absatzbehandlung\newenvironment{absaetze}{\addtolength{\parskip}{\baselineskip}\parindent 0pt}{}%%%%
%\usepackage{geometry}\usepackage{placeins}
\usepackage[font={scriptsize,it},singlelinecheck=false,format=hang,justification=raggedright,skip=15pt]{caption}\usepackage[T1]{fontenc}
\usepackage{ifxetex}\ifxetex %xetex specific stuff \usepackage{xunicode,fontspec,xltxtra}% \usepackage[biolinum]{xelibertine} 
\defaultfontfeatures{Mapping=tex-text} %Fontauswahl% \setmainfont[LetterSpace=1.0,Scale=1]{GaramondNo8} % Hauptzeichensatz% 
\setmonofont{monofur} % Haupt Monospaced %\setsansfont{Kabala} % Haupt ohne Serifen% \setsansfont{URWClassico} % Haupt ohne Serifen %
\setmainfont[LetterSpace=2.0,Scale=.96]{Pali} % Hauptzeichensatz %\setmainfont[LetterSpace=2.0,Scale=1]{Guru} % Hauptzeichensatz %
\setmainfont[LetterSpace=2.0,Scale=1]{jGaramond} % Hauptzeichensatz %\setmainfont[LetterSpace=2.0,Scale=.96]{Liberation Serif} % 
Hauptzeichensatz %\setmainfont[LetterSpace=2.0,Scale=.9]{Bookman Old Style} % Hauptzeichensatz %\setmainfont[LetterSpace=2.0,Scale=1]
{FreeSerif} % Hauptzeichensatz %\setmainfont[LetterSpace=2.0,Scale=.96]{Junicode} % Hauptzeichensatz %\setmainfont[LetterSpace=1.0,Scale=.
96]{MgOpen Cosmetica} % Hauptzeichensatz %\setmainfont[LetterSpace=1.0,Scale=1]{MgOpen Canonica} % Hauptzeichensatz %
\setmainfont[LetterSpace=1.0,Scale=1]{Gentium Book Basic} % Hauptzeichensatz% \setmainfont[LetterSpace=1.0,Scale=1]{Hoefler Text} % 
Hauptzeichensatz %%% %\newfontface\Notationklein[Scale=2]{ScaleDegrees Times} %\newfontface\Notation[Scale=1.4]{ScaleDegrees Times} %
\newfontface\Altdeutsch[Scale=1.1]{Walbaum-Fraktur}% \newfontface\Notenschrift[Scale=1.5]{StaffClefPitchesEasy}% \newfontface\Symbole[Scale=1]
{AppleMyungjo}% \newfontface\Notation[Scale=1.6]{Accidentals}% \newfontface\Notationklein [Scale=2]{Accidentals}% \newfontface
\Altdeutsch[Scale=1.2]{Alte Schwabacher}% \newfontfamily\Chordname[Scale=1]{MgOpen Cosmetica}	% \newfontfamily\Note[Scale=1]{Kabala}\else
\usepackage[utf8]{inputenc} %This can be empty if you are not going to use pdftex %\fi\usepackage[ngerman,portuguese,danish,french,american]
{babel}\usepackage[babel,german=guillemets]{csquotes}\usepackage[normalem]{ulem}\newcommand{\neuq}[1]{\foreignquote{danish}{#1}}
\usepackage{makeidx}\usepackage{calc}%Textpos relevante Sachen\usepackage{textpos}\newcommand{\cd}[2]{\begin{textblock}{2}(#1,#2)                       
\includegraphics[scale=.15]{CD2}                    \end{textblock}            \vspace{4ex}}%\newcommand{\lang}[1]{\widthof{1}(#1)}%\usepackage{MnSymbol}
\usepackage{lettrine}%\usepackage{setspace}%\linespread{1.1}%\onehalfspacing\spacing{1.1}\typearea[current]{last}%%\usepackage{pdfsync} 
Auskommentiert zu Gunsten von synctex%\makeglossary \makeindex\pagestyle{scrheadings}%raggedbottom\usepackage{natbib}
\usepackage{textcomp}%\usepackage{fancyvrb}			% Allow \verbatim et al. in footnotes\usepackage{booktabs}			% Better tables
\usepackage{subfig}\usepackage[subfigure]{tocloft}%\usepackage{array}%\usepackage{tabulary}			% Support longer table cells%
\usepackage{xcolor}				% Allow for color (annotations)%\usepackage[parfill]{parskip}	% Activate to begin paragraphs with an empty	
							% line rather than an indent															%Macros 
für die unterschiedlichen Vorzeichen\newcommand{\Altbe}{ \enquote{\hspace{.5mm}b\hspace{.3mm}}}\newcommand{\Altha}{ \enquote{\hspace{.
5mm}h\hspace{.3mm}}}\newcommand{\nat}{\textsuperscript{\Notationklein \hspace{.3mm}\raisebox{0mm}[0mm][1.9mm]4}}\newcommand{\naturaly}
{\textsuperscript{\Notation \hspace{.3mm}n}}\newcommand{\ben}{\Notation \hspace{.3mm}\raisebox{.9mm}{b}\hspace{.3mm}}\newcommand{\bb}
{\Notation \hspace{.3mm}\raisebox{.9mm}{B}\hspace{.3mm}}\newcommand{\xx}{\Notation \hspace{.3mm}\raisebox{.9mm}{x}\hspace{.3mm}}
\newcommand{\be}{\textsuperscript{\Notation \hspace{.3mm}b}}\newcommand{\beklein}{\textsuperscript{\Notationklein \hspace{.3mm}b}}
\newcommand{\kreuz}{\textsuperscript{\Notation \hspace{.3mm}s}}\newcommand{\kreuzn}{\Notation \hspace{.3mm}\raisebox{.9mm}{s}\hspace{.3mm}}
\newcommand{\kreuzklein}{\textsuperscript{\Notationklein \hspace{.3mm}\raisebox{0mm}[0mm][.5mm]{s}}}\newcommand{\halfdim}
{\textsuperscript{\rmfamily \hspace{.3mm}ø}}%\newcommand{\TRI}{\Symbole ∆}%Macros für die Akkordbezeichnungen\newcommand{\sus}
{\textsuperscript{\Chordname7sus4}}\newcommand{\sechs}{\textsuperscript{\Chordname6}}\newcommand{\sechsneun}{\textsuperscript{\Chordname 
6/9}}\newcommand{\sieben}{\textsuperscript{\Chordname7{\kern -0.15em}}}\newcommand{\siebenflat}{\textsuperscript{\Chordname7({\beklein}5)}}
\newcommand{\siebenflatneunelf}{\textsuperscript{\Chordname7({\beklein}5,9,11)}}\newcommand{\siebensharp}
{\textsuperscript{\Chordname7({\kreuzklein}5)}}\newcommand{\siebenbneun}{\textsuperscript{\Chordname7({\beklein}9)}}
\newcommand{\siebenkneun}{\textsuperscript{\Chordname7({\kreuzklein}9)}}\newcommand{\siebenbneunbdreizehn}
{\textsuperscript{\Chordname7({\beklein}9,{\beklein}13)}}\newcommand{\siebensharpbneun}{\textsuperscript{\Chordname7({\kreuzklein}5,{\beklein}
9)}}\newcommand{\siebenflatbneun}{\textsuperscript{\Chordname7({\beklein}5,{\beklein}9)}}\newcommand{\siebenbneunkelf}
{\textsuperscript{\Chordname7({\beklein}9,{\kreuzklein}11)}}\newcommand{\siebenkneunkelf}{\textsuperscript{\Chordname7({\kreuzklein}9,
{\kreuzklein}11)}}\newcommand{\siebensharpkneun}{\textsuperscript{\Chordname7({\kreuzklein}5,{\kreuzklein}9)}}\newcommand{\siebenkelf}
{\textsuperscript{\Chordname7({\kreuzklein}11)}}\newcommand{\msieben}{\textsuperscript{\Chordname maj7{\kern -0.15em}}}
\newcommand{\msiebensharp}{\textsuperscript{\Chordname maj7({\kreuzklein}5)}}\newcommand{\neun}{\textsuperscript{\Chordname9}}
\newcommand{\neunsharpkelf}{\textsuperscript{\Chordname9({\kreuzklein}5,{\kreuzklein}11)}}\newcommand{\neunsus}
{\textsuperscript{\Chordname9sus4}}\newcommand{\neunkelf}{\textsuperscript{\Chordname9({\kreuzklein}11)}}\newcommand{\neunbdreizehn}
{\textsuperscript{\Chordname9({\beklein}13)}}\newcommand{\mneun}{\textsuperscript{\Chordname maj9{\kern -0.15em}}}
\newcommand{\mneunsharp}{\textsuperscript{\Chordname maj9({\kreuzklein}5)}}\newcommand{\elf}{\textsuperscript{\Chordname11}}
\newcommand{\dreizehnkelf}{\textsuperscript{\Chordname13({\kreuzklein}11)}}\newcommand{\dreizehnbneun}
{\textsuperscript{\Chordname13({\beklein}9)}}\newcommand{\dreizehn}{\textsuperscript{\Chordname13}}\newcommand{\dreizehnnoneun}
{\textsuperscript{\Chordname13(no9)}}\newcommand{\mdreizehn}{\textsuperscript{\Chordname maj13{\kern -0.15em}}}
\newcommand{\mdreizehnkelf}{\textsuperscript{\Chordname maj13({\kreuzklein}11)}}\newcommand{\mdreizehnsharpkelf}
{\textsuperscript{\Chordname maj13({\kreuzklein}5,{\kreuzklein}11)}}\newcommand{\mdreizehnnoelf}{\textsuperscript{\Chordname maj13(no11){\kern 
-0.15em}}}\newcommand{\dreizehnnoelf}{\textsuperscript{\Chordname13(no11)}}\newcommand{\elfnoneun}{\textsuperscript{\Chordname11(no9)}}
\newcommand{\mneunkelf}{\textsuperscript{\Chordname maj9({\kreuzklein}11)}}\newcommand{\msiebenkelf}{\textsuperscript{\Chordname 
maj7({\kreuzklein}11)}}\newcommand{\bedreizehn}{\textsuperscript{\Chordname({\beklein}13)}}\newcommand{\melf}{\textsuperscript{\Chordname 
maj11{\kern -0.15em}}}\newcommand{\dimneun}{°\textsuperscript{\Chordname 7(9)}}% Makros für Abkürzungen\newcommand{\zB}{\mbox{z.\,B. }}
\newcommand{\uU}{\mbox{u.\,U. }}\newcommand{\idr}{\mbox{i.\,d.\,R. }}\newcommand{\iA}{\mbox{i.\,A. }}%\newcommand\myauthor{Author}		
	% In case these were not included in metadata%\newcommand\mytitle{Title}%\newcommand\mykeywords{}%\newcommand
\mybibliostyle{plain}%\newcommand\bibliocommand{}%\newcommand\address{filmmusik@sound-trax.de}\newcommand\myauthor{Jürgen 
Kramlofsky}%\newcommand\baseheaderlevel{2}\newcommand\mycopyright{2008 Jürgen Kramlofsky}%\date{11. November 2008}%\newcommand
\mydate{11. November 2008}%\newcommand\format{complete}\newcommand\mykeywords{Skala, Akkord, Grundton, Terz, Septim, Optionston, 
Optionstöne, Griffbild, Solo, Begleitung}%\newcommand\latexxslt{memoir-twosided.xslt}%\newcommand\language{ngerman}\newcommand
\mytitle{Jazz Akkorde einmal anders}%\newcommand\web{\href{http://www.sound-trax.de}{www.sound-trax.de}}%%	PDF Stuff%
\usepackage[plainpages=false,pdfpagelabels,pdftitle={\mytitle},pagebackref,pdfauthor={\myauthor},pdfkeywords={\mykeywords}]{hyperref}%% Title 
Information%\title{\mytitle}\author{\myauthor}\begin{document}\pagenumbering{roman}%%		Front Matter%\selectlanguage{ngerman}%
\selectlanguage{french}% Title Page\maketitle\clearpage% Copyright Page\vspace*{\fill}\setlength{\parindent}{0pt}\ifx\mycopyright\undefined\else
	\textcopyright{} \mycopyright\fi\setlength{\parindent}{1em}\clearpage% Table of Contents\tableofcontents\listoffigures			% activate to 
include a List of Figures% \listoftables			% activate to include a List of Tables%% Main Content%% Layout settings\setlength{\parindent}
{1em}\mainmatter\pagenumbering{arabic}\chapter{Widmung}\label{widmung}\chapter{Vorwort}\label{vorwort}\lettrine{D}ieses Buch richtet sich in 
erster Linie an den Gitarristen, der keine bzw. wenig Erfahrung mit Jazz Akkorden hat, oder mit der Art und Weise, wie sie ihm bisher vermittelt 
wurden, eher unzufrieden war. Das Buch benutzt einen sehr strukturierten Aufbau in kleinen Schritten. Die Voraussetzungen zum Gebrauch des 
Buches sind lediglich ein rudimentäres Verständnis der Notenschrift (Noten, Notenschlüssel, Takteinteilungen). Die Einfachheit der 
Herangehensweise bedeutet jedoch nicht, dass das Buch nicht auch für den schon fortgeschrittenen Jazz Gitarristen viele neue Ansätze zum 
Verständnis und Erarbeiten neuer Griffbilder oder Akkord-Progressionen bietet.Das erste Kapitel beschäftigt sich in erster Linie mit ein bisschen 
Jazzharmonielehre\index{Jazzharmonielehre}. Hier stehen einige Grundlagen, für all diejenigen, die von Tonleitern\index{Tonleitern} und Akkorden 
noch nichts oder nicht viel gehört haben, oder für jene, die gegen ein bisschen Auffrischung in dieser Richtung nichts einzuwenden haben. Wir 
nehmen die gängigsten Tonleitertypen und bestücken sie sozusagen mit Akkorden. Die hieraus gewonnenen Erkenntnisse setzen wir dann als 
Griffbilder\index{Griffbilder} für die Gitarre um.Alle Beispiel der Akkordfolgen sind auf der beiliegenden CD zum Anhören und Mitspielen 
aufgenommen. Eine Version mit Gitarre um ein Gefühl zu bekommen, wie es klingen könnte und eine Version ohne Gitarre um es selbst zu 
probieren und sich nicht durch eine Vorgabe einschränken zu lassen.chapter{Grundlagen}\label{grundlagen}\section{Einleitung}\label{einleitung}
\lettrine{D}ie in diesem Buch vorkommenden Bezeichnungen und Begriffe entstammen der so genannten amerikanischen Schreibweise für Töne 
und Akkordsymbole\index{Akkordsymbole}, da der Jazz\index{Jazz} eine Musikrichtung ist, welche in den USA entstanden ist und ein Großteil der 
Jazz-Musikliteratur nach wie vor aus Amerika kommt.Konkret bedeutet dies, dass die Töne der C-Dur Tonleiter respektive der natürlichen A-Moll 
Tonleiter nicht den Ton H sondern das B enthalten. Dadurch ergibt sich beginnend auf A folgende Tonreihe: ABCDEFG. Dies entspricht dem Anfang 
des Alphabets. Eigentlich sinnvoller, als AHCDEFG.Eine Anekdote scheint die Erklärung für diesen Umstand zu geben. Noch vor der Erfindung des 
Buchdrucks lag die einzige Möglichkeit ein Buch oder Schriftstück zu vervielfältigen darin, es abzuschreiben. Selbst als der Buchdruck schon 
erfunden war, wurde diese Methode noch eine ganze Zeit weitergeführt. Dies geschah fast ausschließlich in den damaligen Klöstern. Der 
Erzählung nach hat sich also ein Kopist eines deutschen Klosters bei der Abschrift eines Schriftstückes der Musiklehre ein wenig vertan, bzw. nicht 
ganz genau hingesehen. So hat er beim kopieren aus dem altdeutschen Buchstaben {\Altbe} aus Versehen ein altdeutsches {\Altha} gemacht. Schon 
war es passiert; wenn man der Geschichte glauben schenken mag.Eine weitere Abweichung bei der Bezeichnung der Töne tritt dann ein, wenn 
diese Töne verändert, also alteriert werden. Alterationen\index{Alterationen} (Erhöhungen oder Erniedrigungen) dieser sieben Töne werden 
nicht, wie in Deutschland üblich durch das Anhängen der Silben -es (erniedrigt) bzw. -is (erhöht) kenntlich gemacht, sonder durch Anhängen der 
entsprechenden Vorzeichen aus der Notenschrift an den Notenbuchstaben. Aus einem erhöhten C wird dann also kein Cis sondern ein C\kreuz. Aus 
einem erniedrigtem D kein Des sondern ein D\be. Die englische Aussprache dieser beiden Töne ist dann C~sharp bzw. D~flat. Es bleibt einem 
natrlich selbst berlassen, trotzdem weiterhin die deutsche Aussprache zu verwenden.Unser deutsches B wird also in der englischen Schreibweise als 
B\be\ geschrieben. Damit hier keine Ungereimtheiten oder Verwechslungen entstehen knnen, werden kleine Hilfen hinzugesetzt. Das deutsche H 
wird also als B\nat\ (engl. B natural), das deutsche B als B\be\ (B~flat) geschrieben.Dieses Buch ist mehr oder weniger in 3 Abschnitte unterteilt. 
\index{Basisakkorde} Basisakkorde, Basisakkorde plus Optionstöne und Akkorde ohne Grundtöne. Die verwendeten Bezeichnungstypen für die 
jeweiligen Akkorde zeige ich hier an Hand von Akkorden mit dem Grundton C. Der große C-Durseptakkord wird in diesem Buch als C\msieben\ 
geschrieben, C-Dominatseptakkord als C\sieben, der C-Mollseptakkord als Cm\sieben, der halbverminderte C-Mollseptakkord als Cm\siebenflat\ und 
der verminderte C-Septakkord als C°\sieben. Im Englischen wird der verminderte Septakkord als \neuq{diminished} bezeichnet. Deswegen wird in 
diesem Buch des öfteren die Bezeichnung Dim\sieben\ für diesen Akkord verwendet.Für den maj7\index{maj7} Akkord findet man durchaus auch 
viele andere Abkürzungen, wie z.B. MAJ7, Maj7, M7, j7, ∆7 etc.\\Für den m\sieben\ Akkord Abkürzungen wie: min7, -7, mi7 etc.\\Für den m
\siebenflat\ Akkord die Bezeichnung: \halfdim7 --- Im Zusammenhang mit diesem Symbol hört man oft auch die Bezeichnung \neuq{halbverminderter 
Septakkord}. Diese Benennung ist allerdings nicht ganz korrekt.Was diese ganzen Zahlen und Buchstabenkombinationen bedeuten, wird aber hier 
im ersten Kapitel noch einmal gründlich an Hand unterschiedlicher Beispiele erklärt.\clearpage\noindentUnd nun  A-Melodisch Moll.\vspace{6mm}
{%
\parindent 0pt
\noindent
\ifx\preLilyPondExample \undefined
\else
  \expandafter\preLilyPondExample
\fi
\def\lilypondbook{}%
\input bf/lily-41ff04e9-systems.tex
\ifx\postLilyPondExample \undefined
\else
  \expandafter\postLilyPondExample
\fi
}{\Chordname  \hspace{5mm} Am\mdreizehn\ \hfill\ Bm\dreizehnnoneun\ \hfill\ C\mdreizehnsharpkelf\ \hfill\ D\dreizehn\hfill\ E
\neunbdreizehn\ \hfill\ F{\kreuz}m\siebenflatneunelf\hfill\ G{\kreuz}m\siebenflat\hfill}\vspace{6mm}%% Back Matter%%\chapter{Anhang}%
\label{anhang}%\appendixpage%\appendix\backmatter%	Bibliography%\bibliographystyle{\mybibliostyle}%\bibliocommand%	Glossary%
\printglossary%	Index%\renewcommand{\indexname}{Sachregister} \addcontentsline{toc}{section}{Index}\printindex\end{document}