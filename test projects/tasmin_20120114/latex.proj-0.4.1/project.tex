% 
% This document is copyrighted by Robert G. Brown as of the latest 
% revision date below, as are all earlier revisions.
%
% $Id: project.tex 37 2006-10-26 06:54:23Z rgb $
%
% Filename: project.tex
% Description:  Latex sources for a simple article/project

\documentclass{article}
\usepackage{epsfig}
\usepackage{html}

% Useful parameters to set in the book documentclass.
% \setlength{\topmargin}{-1.0in}
% \setlength{\textheight}{9.2in}
% \setlength{\oddsidemargin}{0.0in}
% \setlength{\textwidth}{6.5in}
%   EITHER comment both out OR uncomment both of these, usually.
%   As it is, it doesn't indent but skips a small space between
%   paragraphs, good for e.g. lecture notes, not so good for books
%   or articles depending.
\setlength{\parindent}{0.0in}
\setlength{\parskip}{0.1in}

\newcommand{\martin}[1]{\textcolor{blue}{\textit{Martin: #1}}}
\newcommand{\miquel}[1]{\textcolor{red}{\textit{Miquel: #1}}}
\newcommand{\christian}[1]{\textcolor{green}{\textit{Ch.: #1}}}

\newcommand{\etc}{\textit{etc}}
\newcommand{\ie}{\textit{i.e.}}
\newcommand{\da}{\textsc{LTPDA}}
\newcommand{\code}[1]{\texttt{#1}}
\newcommand{\tbd}[0]{\textcolor{red}{TBD}}
\newcommand{\ltpda}[0]{\textsc{LTPDA}}

\newcommand{\note}[1]{\textcolor{blue}{(#1)}}




% Spare figure template
%\begin{figure}
%  \centerline{
%    \def \epsfsize#1#2{0.5#1}
%    \centerline{\epsfbox{vector.1.eps}}
%  }
%  \caption{This would be a figure.}
%  \label{figure_label}
%\end{figure}
%
% Inline unnumbered figure (no caption)
% \def \epsfsize#1#2{0.5#1}
% \centerline{\epsfbox{vector.1.eps}}
% \medskip
% 
%

\begin{document}

\title{A Nifty Latex Template}

\date{\today}

\author{\bf Robert G. Brown \\
Duke University Physics Department \\
Durham, NC 27708-0305 \\
rgb@phy.duke.edu}

\maketitle

\vspace*{\fill}

\centerline{\large \bf Copyright Notice}
\centerline{Copyright Robert G. Brown 2006}
\centerline{See attached Open Publication License}

\newpage

\tableofcontents

\newpage

% Inline figure example
\def \epsfsize#1#2{0.5#1}
\centerline{\epsfbox{project.eps}}

% For all books and most complex documents (with many sections) it is
% probably a good idea to break up the source into many smaller files, one
% per chapter or section.  You could then input them like:
% \input{section_one.tex} or \input{chapter_one.tex}
% This document is very simple, so I don't bother, except for the OPL
% which is "standard" to all my documents of this sort.  If you don't
% like it or need it, comment it out or delete it and remove the file.
\section{Section One}

This is text in section one.

It can contain inline math: $E = \hbar \omega$.

It can contain numbered equations:
\be
  H\psi = E\psi = \hbar \omega \psi
\ee

It can even contain numbered aligned equation ``arrays'':
\bea
 H\psi & = & -i\hbar \partialdiv{\psi}{t} \nonumber \\
       & = & \hbar \omega \psi \nonumber \\
       & = & E \psi
\eea


\subsection{Subsectioning Text}

Subsectioning text is certainly possible, and yields a trace in the
table of contents (if any).

\subsubsection{Subsubsectioned Text}

Text at this level still gets a TOC entry in some document classes, but
not all.  TOC's usually only go three levels deep, four if you use
the ``part'' command in a book.  Still useful for the formatting.

That's {\em really} about it.  You can do stuff like {\bf boldfacing}
and using {\tiny tiny} fonts or {\huge HUGE} fonts, but you probably
shouldn't as they make the text look odd.  You can do tables and tabs
and much more, but that is too much to demo in a simple template like
this, especially when a lot of that will vary as you add packages.

Simple latex is the best, unless you are writing for a very specific
purpose with very specific requirements.  It is what latex is really
designed for -- the whole idea of markup is to trust {\em professionals}
to lay out appropriate fonts, sizes, and so on for various document
objects in a completely uniform way.  Latex documents invariably look
like they are ripped right out of the pages of a book, even when they
are really simple ones (like this one)!

Good Luck!

The following appendix is an {\em example} of a Gnu Open Publication
License.  Don't worry, you can take this template and use it any way
that you wish, with or without the OPL announcement.  However, if you
use it you owe me a beer...

\input{OPL.tex}

\end{document}



